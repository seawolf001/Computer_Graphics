\documentclass[10pt,a4paper]{article}
\usepackage[utf8]{inputenc}
\usepackage[frenchb]{babel}
\usepackage[OT1]{fontenc}
\usepackage{amsfonts, amsmath, amssymb, amsthm, dsfont, amsthm}
\usepackage{a4wide}
\usepackage[dvipsnames]{xcolor}
\usepackage{tikz} 
\usetikzlibrary{arrows,positioning,shapes}

\title{\textbf{Lab Report} \\ Week of 15/01/2017}
%\author{Olivier \textsc{Mangin}}
%\date{\today}

\definecolor{main}{named}{BurntOrange}
\definecolor{second}{named}{RoyalBlue}
%\newcommand{\maincolor}{orange}
%\newcommand{\secondcolor}{orange!20}
\newcommand{\strong}[1]{\textcolor{main}{\textbf{#1}}}
\newcommand{\stronger}[1]{\textcolor{second}{\textbf{#1}}}
\newcommand{\colored}[1]{\textcolor{main}{#1}}

% Affichage du titre avec les numéro et date de la semaine
\newcommand{\titre}[2]{
\noindent
\hspace{-10pt}
\begin{tabular}{lr}
  \hspace{0.58\textwidth} & \hspace{0.4\textwidth} \\
  \strong{\huge Lab Report} & \textbf{\Large #1} \medskip \\
  \textbf{\Large Jitendra Kumar , 1401CS19} & {\large #2} ~\\
\end{tabular}

\vspace{20pt}
}

% Encadré ``En bref'' réumant les avancées et problèmes de la semaine
\newenvironment{enbref}{
\noindent\fcolorbox{main}{main}{
\begin{minipage}{\textwidth}
\textcolor{white}{\textbf{\large }}
\end{minipage}
} \\

}{
\begin{center}
  \strong{ \rule[2mm]{\textwidth}{3pt} }
\end{center}
\vspace*{-20pt}
}

% Affichage d'un titre de rubrique
\newcommand{\rubrique}[1]{
  \bigskip
  \begin{center}
  \begin{minipage}{\textwidth}
    \noindent\strong{{\large #1} \\
      \rule[2mm]{\textwidth}{1pt} }
  \end{minipage}
  \end{center}
  \vspace*{-20pt}
}

% Symbole utilisé en début de ligne des éléments
\newcommand{\doublerect}{
\begin{tikzpicture}
  \fill[color=main] (0,0) rectangle (4pt,-4pt);
  \fill[color=second] (2pt,-2pt) rectangle (6pt,-6pt);
\end{tikzpicture}
}

% Affichage d'un titre d'élément
\newcommand{\element}[1]{
  \medskip
  \noindent\textcolor{second}{ \doublerect \textbf{#1}}
}

% Pour les lectures, petit raccourci pour mettre en avant le niveau
% de lecture d'un article.
\newcommand{\lu}{\strong{[Lu]} }
\newcommand{\parcouru}{\strong{[Parcouru]} }
\newcommand{\alire}{\strong{[A lire]} }
\newcommand{\presentation}{\strong{[Présentation]} }
\newcommand{\keynote}{\strong{[Keynote]} }
